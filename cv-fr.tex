%!TEX TS-program = xelatex
\documentclass[]{friggeri-cv}

\usepackage[french]{babel}
\usepackage{qrcode}
\usepackage{xifthen}
\usepackage[heightr=0.4,roundnessr=0,subdivisions=1,borderwidth=0pt,width=1cm]{progressbar}

\patchcmd{\aside}
{\begin{textblock}{3.6}(1.5, 4.33)}
	{\begin{textblock}{5.3}(0, 4.33)}
		{}{}

\begin{document}
\header{baptiste }{candellier}
{ingénieur en informatique}

% In the aside, each new line forces a line break
\begin{aside}
    \section{contact}
    40 quai Vendeuvre
    14000 Caen
    France
    ~
    \href{tel:0033641666355}{+33 6 41 66 63 55}
    \href{mailto:baptiste.candellier@gmail.com}{baptiste.candellier@gmail.com}
    ~
    {\NoAutoSpacing\href{https://dev.outadoc.fr/fr}{dev.outadoc.fr/fr}}
    {\NoAutoSpacing\href{https://www.linkedin.com/in/candellierba/}{linkedin.com/in/candellierba}}
    \section{langues}
    français natif
    anglais courant (TOEIC : 990)
    notions d'espagnol
    \section{programmation}
    Kotlin + Android \progressbar {1.0}
    Python 3 \progressbar {0.7}
    \csharp{} .NET \progressbar {0.9}
    Java \progressbar {0.8}
    HTML / CSS3 /JS \progressbar {0.8}
    PHP + Symfony \progressbar {0.8}
    C \progressbar {0.7}
    C++ \progressbar {0.6}
    Scala \progressbar {0.5}
    Arduino \progressbar {0.4}
\end{aside}

\section{expérience}

\begin{entrylist}
	\entry
	{2015--ajd.}
	{\href{https://www.natixis.com}{Natixis}}
	{Apprentissage en alternance, Caen}
	{Apprentissage de 3 ans avec l’ENSICAEN. Développement \csharp .NET et Java dans une équipe agile pour la conception de logiciels internes. Stage de 3 mois dans la branche de Porto, Portugal.}
	{Telerik, API .NET Core, UX design, LDAP, IBM FileNet, TDD et BDD, Spring}
	
	\entry
	{avr.--juin\\2015}
	{\href{https://www.nxp.com}{NXP Semiconductors}}
	{Stage, Colombelles}
	{Développement d’un module Java EE pour la connection à un annuaire Active Directory, dans une petite équipe en mode agile.}
	{LDAP, Java EE, scrum, agilité, Apache Tomcat}
\end{entrylist}

\section{éducation}

\begin{entrylist}
    \entry
    {2015--2018}
    {Diplôme d’Ingénieur {\normalfont en Informatique}}
    {ENSICAEN, Caen}
    {Apprentissage en informatique, spécialité monétique et sécurité.}{}
    
    \entry
    {2013--2015}
    {D.U.T {\normalfont Informatique}}
    {Université de Caen}
    {Bases de données, C, Java, PHP, économie, communication...}{}
    
    \entry
    {2013}
    {Baccalauréat {\normalfont Scientifique}}
    {Lycée Charles de Gaulle, Caen}
    {Mention bien, spécialité SVT.}{}
\end{entrylist}

\section{projets}

\begin{entrylist}
	\entry
	{2017}
	{Compteur de vitesse numérique}
	{\href{https://github.com/outadoc/bttf-speedometer-arduino}{github.com/outadoc/bttf-speedometer-arduino}}
	{Réplique du compteur de vitesse de Retour vers le Futur. Connexion OBD-II avec l'ordinateur de bord, liaison série avec un Arduino.}
	{C++, électronique, Arduino, OBD-II}
	
	\entry
	{2016}
	{Linkindle}
	{\href{https://dev.outadoc.fr/fr/projects/linkindle}{dev.outadoc.fr/fr/projects/linkindle}}
	{Affichage du graphe de consommation énergétique directement depuis un compteur Linky, utilisant une Kindle comme afficheur déporté, sans capteur supplémentaire.}
	{Python, web scraping, API reverse-engineering, NGINX}
    
    \entry
    {2013}
    {Twistoast}
    {\href{https://dev.outadoc.fr/fr/projects/twistoast}{dev.outadoc.fr/fr/projects/twistoast}}
    {Une application Android pour suivre les horaires en temps réel des bus et trams de Caen, avec notifications et application pour smartwatch.}
    {Kotlin, SQLite, web scraping, API reverse-engineering, TDD et BDD}
    
%    \entry
%    {2012}
%    {SkinSwitch {\normalfont for Minecraft}}
%    {\href{https://dev.outadoc.fr/fr/projects/skinswitch}{dev.outadoc.fr/fr/projects/skinswitch}}
%    {Gestionnaire de skins pour Minecraft. Version iOS, puis Android, avec support d'Android Wear.}
%    {Java, SQLite, web scraping, API reverse-engineering}
\end{entrylist}

\section{intérêts}

smartphones, wearables, internet des objets (IoT), design, cartes à puce, transactions sans contact, sécurité...
...rock et ciné

\end{document}

%!TEX TS-program = xelatex
\documentclass[]{friggeri-cv}

\usepackage[french]{babel}

\patchcmd{\aside}
{\begin{textblock}{3.6}(1.5, 4.33)}
	{\begin{textblock}{5.3}(0, 4.33)}
		{}{}

\begin{document}
\header{baptiste }{candellier}
{ingénieur en informatique}

% In the aside, each new line forces a line break
\begin{aside}
    \section{contact}
    40 quai Vendeuvre
    14000 Caen
    France
    ~
    \href{tel:0033641666355}{+33 6 41 66 63 55}
    \href{mailto:baptiste.candellier@gmail.com}{baptiste.candellier@gmail.com}
    ~
    {\NoAutoSpacing\href{https://dev.outadoc.fr/fr}{dev.outadoc.fr/fr}}
    {\NoAutoSpacing\href{https://www.linkedin.com/in/candellierba/}{linkedin.com/in/candellierba}}
    \section{langues}
    français natif
    anglais courant
    notions d'espagnol
    \section{programmation}
    Kotlin + Android
    Java, J2E
    \csharp .NET (WPF, Core)
    C / C++
    PHP + Symfony
    HTML / CSS3 /JS
    Scala
\end{aside}

\section{expérience}

\begin{entrylist}
	\entry
	{sep. 2015\\--août 2018}
	{\href{https://www.natixis.com}{Natixis}}
	{Apprentissage en alternance, Caen}
	{Apprentissage de 3 ans avec l’ENSICAEN. Développement \csharp .NET et Java dans une équipe agile pour la conception de logiciels internes. Telerik, API .NET Core, UX design, LDAP, IBM FileNet, TDD et BDD, Spring-batch.}
	\entry
	{avr.\\--juin 2015}
	{\href{https://www.nxp.com}{NXP Semiconductors}}
	{Stage, Colombelles}
	{Développement d’un module Java (Jakarta) EE pour la connection à un annuaire LDAP Active Directory, dans une petite équipe en mode agile.}
\end{entrylist}

\section{éducation}

\begin{entrylist}
    \entry
    {2015--2018}
    {Diplôme d’Ingénieur {\normalfont en Informatique}}
    {ENSICAEN, Caen}
    {Apprentissage en informatique, spécialité monétique et sécurité.}
    \entry
    {2013--2015}
    {D.U.T {\normalfont Informatique}}
    {Université de Caen}
    {Bases de données, C, Java, PHP, économie, communication...}
    \entry
    {2013}
    {Baccalauréat {\normalfont Scientifique}}
    {Lycée Charles de Gaulle, Caen}
    {Mention bien, spécialité SVT.}
\end{entrylist}

\section{projets}

\begin{entrylist}
	\entry
	{2016}
	{Linkindle}
	{\href{https://dev.outadoc.fr/fr/projects/linkindle}{dev.outadoc.fr/fr/projects/linkindle}}
	{Affichage du graphe de consommation énergétique directement depuis un compteur Linky, utilisant une Kindle comme afficheur déporté, sans capteur supplémentaire.
	\emph{Python, web scraping, API reverse-engineering, nginx}}
    \entry
    {2013}
    {Twistoast}
    {\href{https://dev.outadoc.fr/fr/projects/twistoast}{dev.outadoc.fr/fr/projects/twistoast}}
    {Une application pour suivre les horaires en temps réel des bus et trams de Caen, avec notifications et application pour smartwatch.
    \emph{Kotlin, SQLite, web scraping, API reverse-engineering, behaviour-driven development.}}
    \entry
    {2012}
    {SkinSwitch for Minecraft}
    {\href{https://dev.outadoc.fr/fr/projects/skinswitch}{dev.outadoc.fr/fr/projects/skinswitch}}
    {Gestionnaire de skins pour Minecraft. Version iOS, puis Android, avec support d'Android Wear.
    \emph{Java, SQLite, web scraping, API reverse-engineering}}
\end{entrylist}

\section{intérêts}

smartphones, wearables, internet des objets (IoT), design, cartes à puce, transactions sans contact, sécurité...
...rock et ciné

\end{document}

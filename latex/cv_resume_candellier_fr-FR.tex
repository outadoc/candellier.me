% !TEX TS-program = xelatex
% !TEX encoding = UTF-8
% !TEX spellcheck = fr-FR

\documentclass[]{friggeri-cv}

\usepackage[french]{babel}
\usepackage{csquotes}
\usepackage{calc}
\usepackage{xifthen}
\usepackage[heightr=0.4,roundnessr=0,subdivisions=1,borderwidth=0pt,width=1cm]{progressbar}

\usetikzlibrary{calc}

\patchcmd{\aside}
{\begin{textblock}{3.6}(1.5, 4.33)}
    {\begin{textblock}{5.3}(0, 4.33)}
        {}{}

\begin{document}
\header{Baptiste }{Candellier}
{développeur Android junior}

% In the aside, each new line forces a line break
\begin{aside}
    \section{profil}
    25 ans
    Nationalité française
    Permis B et véhicule
    Basé Lyon 7
    ~
    \href{tel:0033641666355}{+33 6 41 66 63 55}
    \href{mailto:baptiste.candellier@gmail.com}{baptiste.candellier@gmail.com}
    {\NoAutoSpacing\href{https://baptiste.candellier.me}{baptiste.candellier.me}}
    {\NoAutoSpacing\href{https://www.linkedin.com/in/candellierba/}{linkedin.com/in/candellierba}}
    \section{langues}
    français natif
    anglais courant
    notions d'espagnol
    \section{programmation}
    Kotlin (Android; Multiplatform)
    Java (Android; server-side)
\end{aside}

\section{expérience}

\begin{entrylist}
    \entry
    {jan. 2018--adj.}
    {\href{https://www.tapptic.com}{Tapptic} \textnormal{pour} \href{https://www.bedrockstreaming.com}{Bedrock} (anciennement M6 Web)}
    {Lyon (69)}
    {Maintenance et développement des applications Android d'AVOD \emph{6play (France)}, \emph{RTLplay (Belgique)}, \emph{RTL Most (Hongrie)}, \emph{RTLplay (Croatie)}. \\Expansion de cette plateforme avec une refonte majeure pour le développement de l'application de SVOD \emph{Salto} et futurs clients.}
    {Android, Kotlin, Java, RxJava, Toothpick, Retrofit, Picasso, Bitrise, diverses plateformes d'analytics, Terraform + AWS}

    \entry
    {sep. 2018--déc. 2018}
    {\href{https://www.keyconsulting.fr}{Key Consulting} \textnormal{pour} \href{https://www.natixis.com}{Natixis}}
    {Caen (14)}
    {Continuation du projet facturation de Natixis en tant que prestataire.}{}

    \entry
    {sep. 2015--aoû. 2018}
    {\href{https://www.natixis.com}{Natixis} \textnormal{(Apprentissage en alternance)}}
    {Caen (14)}
    {Apprentissage de 3 ans avec l’ENSICAEN. Développement \csharp .NET et Java dans une équipe agile pour la conception de logiciels internes.\\Stage de 3 mois dans la branche de Porto, Portugal.}
    {Telerik, API .NET Core, UX design, LDAP, IBM FileNet, TDD/BDD, Spring}

    \entry
    {avr.--juin\\2015}
    {\href{https://www.nxp.com}{NXP Semiconductors} \textnormal{(Stage)}}
    {Colombelles (14)}
    {Développement d’un module Java EE pour la connection à un annuaire Active Directory, dans une petite équipe en mode agile.}
    {LDAP, Java EE, kanban, Apache Tomcat, Inkscape}
\end{entrylist}

\section{éducation}

\begin{entrylist}
    \entry
    {2015--2018}
    {Diplôme d’Ingénieur {\normalfont en Informatique}}
    {ENSICAEN, Caen (14)}
    {Spécialité monétique et sécurité. Apprentissage de 3 ans.}{}

    \entry
    {2013--2015}
    {D.U.T {\normalfont Informatique}}
    {Université de Caen (14)}
    {Bases de données, C, Java, PHP, économie, communication...}{}

    \entry
    {2013}
    {Baccalauréat {\normalfont Scientifique}}
    {Lycée Charles de Gaulle, Caen (14)}
    {Mention bien, spécialité SVT.}{}
\end{entrylist}

\section{projets}

\begin{entrylist}
    \entry
    {2020--ajd.}
    {Home Slide pour Home Assistant}
    {\href{https://github.com/outadoc/home-slide-android}{github.com/outadoc/home-slide-android}}
    {Application Android libre et open-source permettant de contrôler une plateforme domotique (Home Asssitant).}
    {Kotlin, coroutines, view models, Koin, Wear OS, flow OAuth2}

    \entry
    {2020}
    {Interpréteur CHIP-8}
    {\href{https://github.com/outadoc/kemu-chip8}{github.com/outadoc/kemu-chip8}}
    {Projet de découverte de Kotlin/Multiplatform. Un interpréteur (émulateur) pour CHIP-8, permettant de charger des programmes, écrit en pur Kotlin, avec clients Java et JS.}
    {Kotlin, coroutines, Kotlin/Multiplatform}

\end{entrylist}

\end{document}

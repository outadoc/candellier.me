% !TEX TS-program = xelatex
% !TEX encoding = UTF-8
% !TEX spellcheck = fr-FR

\documentclass[]{friggeri-cv}

\usepackage[french]{babel}
\usepackage{csquotes}
\usepackage{calc}
\usepackage{xifthen}
\usepackage[heightr=0.4,roundnessr=0,subdivisions=1,borderwidth=0pt,width=1cm]{progressbar}

\usetikzlibrary{calc}

\patchcmd{\aside}
{\begin{textblock}{3.6}(1.5, 4.33)}
	{\begin{textblock}{5.3}(0, 4.33)}
		{}{}

\begin{document}
\header{baptiste }{candellier}
{ingénieur en informatique}

% In the aside, each new line forces a line break
\begin{aside}
    \section{profil}
    40 quai Vendeuvre
    14000 Caen
    France
    ~
    Nationalité française
    23 ans (juillet 1995)
    Permis B et véhicule
    ~
    \href{tel:0033641666355}{+33 6 41 66 63 55}
    \href{mailto:baptiste.candellier@gmail.com}{baptiste.candellier@gmail.com}
    ~
    {\NoAutoSpacing\href{https://dev.outadoc.fr/fr}{dev.outadoc.fr/fr}}
    {\NoAutoSpacing\href{https://www.linkedin.com/in/candellierba/}{linkedin.com/in/candellierba}}
    \section{langues}
    français natif
    anglais courant (TOEIC 990)
    notions d'espagnol
    \section{programmation}
    Kotlin
    Java
    Android
    \csharp{} (WPF, ASP.NET Core)
    \restorecr 
    % That's the only place where this could fit without taking up a whole paragraph on the page...
    \begin{tikzpicture}[remember picture, overlay]
	    \path let \p1 = (current page.north east) in node[anchor=north east] at (\x1-1.2cm,\y1-0.8cm) {\includegraphics[height=2.2cm]{qrcode}}; 
    \end{tikzpicture}
    \obeycr
\end{aside}

\section{expérience}

\begin{entrylist}
	\entry
	{sep. 2018--ajd.}
	{\href{https://www.natixis.com}{Natixis} \normalfont \emph{via} \href{https://www.keyconsulting.fr}{Key Consulting}}
	{Caen}
	{Continuation du projet facturation de Natixis en tant que prestataire.}{}
	
	\entry
	{sep. 2015--aoû. 2018}
	{\href{https://www.natixis.com}{Natixis}}
	{Apprentissage en alternance, Caen}
	{Apprentissage de 3 ans avec l’ENSICAEN. Développement \csharp .NET et Java dans une équipe agile pour la conception de logiciels internes.\\Stage de 3 mois dans la branche de Porto, Portugal.}
	{Telerik, API .NET Core, UX design, LDAP, IBM FileNet, TDD/BDD, Spring}
	
	\entry
	{avr.--juin\\2015}
	{\href{https://www.nxp.com}{NXP Semiconductors}}
	{Stage, Colombelles}
	{Développement d’un module Java EE pour la connection à un annuaire Active Directory, dans une petite équipe en mode agile.}
	{LDAP, Java EE, kanban, Apache Tomcat, Inkscape}
\end{entrylist}

\section{éducation}

\begin{entrylist}
    \entry
    {2015--2018}
    {Diplôme d’Ingénieur {\normalfont en Informatique}}
    {ENSICAEN, Caen}
    {Spécialité monétique et sécurité. Apprentissage de 3 ans.}{}
    
    \entry
    {2013--2015}
    {D.U.T {\normalfont Informatique}}
    {Université de Caen}
    {Bases de données, C, Java, PHP, économie, communication...}{}
    
    \entry
    {2013}
    {Baccalauréat {\normalfont Scientifique}}
    {Lycée Charles de Gaulle, Caen}
    {Mention bien, spécialité SVT.}{}
\end{entrylist}

\section{projets}

\begin{entrylist}
	 \entry
	{2013--ajd.}
	{Twistoast}
	{\href{https://dev.outadoc.fr/fr/projects/twistoast}{dev.outadoc.fr/fr/projects/twistoast}}
	{Application Android pour suivre les horaires en temps réel des bus et trams de Caen, avec notifications et app pour smartwatch.}
	{Kotlin, SQLite, web scraping, API reverse-engineering, TDD et BDD}
	
	\entry
	{2018}
	{Borne 2 Be}
	{\href{https://dev.outadoc.fr/fr/projects/borne2be}{dev.outadoc.fr/fr/projects/borne2be}}
	{Réalisation d'un kioske communautaire de quartier, avec authentification par badge sans contact, dans une équipe de 13 personnes diverses.}
	{Android Things, NFC, Apache Cordova, Raspberry Pi, impression 3D}
	
	\entry
	{2016}
	{Linkindle}
	{\href{https://dev.outadoc.fr/fr/projects/linkindle}{dev.outadoc.fr/fr/projects/linkindle}}
	{Affichage du graphe de consommation énergétique directement depuis un compteur Linky, utilisant une Kindle comme afficheur déporté.}
	{Python, web scraping, API reverse-engineering, NGINX}
	
 	\entry
	{2015}
	{SkinSwitch}
	{\href{https://dev.outadoc.fr/fr/projects/skinswitch}{dev.outadoc.fr/fr/projects/skinswitch}}
	{Gérez facilement vos skins Minecraft depuis votre téléphone. Représenta ma première expérience avec Material et support d'Android Wear 1.0.}
	{Android, gradle, API reverse-engineering, design, Android Wear}
    
\end{entrylist}

\section{intérêts}

wearables, internet des objets (IoT), domotique, smartphones, design, cartes à puce, nfc, transactions sans contact, vie privée, sécurité, blockchain, rock, photographie

\end{document}
